\documentclass[12pt, a4paper, oneside]{ctexart}
\usepackage{amsmath, amsthm, amssymb, bm, graphicx, hyperref, mathrsfs}

\title{\textbf{Homework 1: Environment}}
\date{\today}
\linespread{1.5}
\begin{document}

\maketitle


\section{计算机信息}
\begin{itemize}
    \item \textbf{型号:} 82JD Lenovo Legion Y9000P2021H 
    \item \textbf{内存:} 16GB
    \item \textbf{硬盘:} 512GB SSD + 512GB SSD
    \item \textbf{CPU:} 11th Gen Intel i7-11800H (16) @ 4.600GHz 
    \item \textbf{GPU:} NVIDIA GeForce RTX 3060 Mobile / Max-Q, Intel TigerLake-H GT1 [UHD Graphics] 
    
\end{itemize}
\section{Linux实现方式、版本以及安装的软件}
\begin{itemize}
    \item \textbf{双系统:} Windows 10 + Kubuntu 22.04.2 LTS 
    \item \textbf{安装的软件:} Edge, Firefox, Google, WPS, LibreOffice, Zotero, Typora, Clash for Windows, Mathpix, LX Music,
GeoGebra, paraview, MATLAB, Pycharm, libdealii, etc. 
    
\end{itemize}
\section{编辑器和gcc版本}
\begin{itemize}
    \item \textbf{编辑器:} VS Code 1.79.2
    \item \textbf{gcc版本:} gcc (Ubuntu 11.3.0-1ubuntu1~22.04.1) 11.3.0

\end{itemize}

\section{使用Linux环境工作的可能性和可能场景}
\begin{itemize}
    \item \textbf{可能性:} 比较大,我本身专业是电子科学与技术,很多专业软件只有Linux版本或者在Linux环境下运行效率高。
    \item \textbf{可能场景:}集成电路设计,使用Linux环境下的EDA软件如Synopsys、Cadence等进行集成电路的布局、仿真、验证等操作;科学计算,使用Linux环境强大的计算能力和丰富的科学计算库进行电路设计、信号处理和系统建模等工作;嵌入式系统开发,使用Linux环境进行驱动程序开发、嵌入式应用程序编程和硬件/软件集成等工作。

\end{itemize}


\end{document}